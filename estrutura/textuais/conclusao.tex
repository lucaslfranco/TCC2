\chapter{CONCLUSÃO}
\label{chap:conclusao}

O desenvolvimento deste trabalho possibilitou compreender o impacto que a retenção acadêmica possui para as universidades brasileiras, e refletir sobre quais são as formas de contribuir para a redução este problema. 
A retenção acadêmica é uma das variáveis relacionadas à evasão acadêmica e uma das formas de se iniciar a trabalhar sobre a evasão é observar anteriormente o fenômeno da retenção acadêmica.

Foi-se então implementado um ``ecossistema'' de análise de dados acadêmicos para o fornecimento de mecanismos que permitam atuar sobre a retenção acadêmica e estimar o status final dos alunos.

A ferramenta desenvolvida neste trabalho possibilita que docentes tenham uma maior aplicação dos dados que possuem, permitindo que estes carreguem dados das disciplinas ministradas e a partir disso utilize mecanismos para comunicar-se com alunos em perfil tendencioso à retenção, a fim de compreender e impedir seu acontecimento.

Com o estudo realizado sobre os dados de alunos da UTFPR - Campus Cornélio Procópio foi possível implementar modelos preditivos que atuam a fim de antecipar a retenção acadêmica ocorrida principalmente devido a ausências e assim garantir que os docentes engajados possam realizar tomadas de decisão para auxiliar os alunos na regularização das suas situações. 

Dentre os algoritmos de mineração de dados empregados o \textit{Gradient Boosting} se mostrou mais adequado para o conjunto de dados utilizado, com uma acurácia obtida entre 89,8\% e 97,4\%.

Como oportunidades de trabalhos futuros tem-se a exploração em mais detalhes de dados dos alunos, dentre esses os dados sociais, como a idade, gênero, histórico familiar e educacional, e os de cunho financeiro como a renda familiar e o recebimento de bolsas de auxílio.
Com estes dados é possível ampliar o objeto de estudo deste trabalho para além da retenção, alcançando também o âmbito da evasão acadêmica, que possui uma natureza muito mais subjetiva.

Outra possibilidade de trabalho é análise de diferentes tipos de cursos, comparando por exemplo cursos presenciais e de educação à distância, no qual a frequência não é um dos fatores chave para a aprovação.

O acesso de assistentes psicopedagógicos na ferramenta também é uma possibilidade de expansão, desta forma profissionais com um \textit{background} mais específico na interação com os alunos poderiam realizar um trabalho mais direcionado aos alunos que foram identificados com risco de retenção.