\chapter{ANÁLISE DOS RESULTADOS}

\section{MODELOS}

\subsection{Acurácia}

Foram criados dois modelos com cada algoritmo de \textit{machine learning} testado, o primeiro considerando somente uma avaliação e o segundo com notas de duas avaliações.
Os algoritmos utilizados foram o \textit{Gradient Boosting}, \textit{Random Forest} e de Regressão Linear, já descritos anteriormente, e ao realizar o treinamento dos modelos criados com o conjunto de dados disponível foram obtidos seus \textit{scores}, ou acurácias, com o auxilio da biblioteca \textit{Scikit-learn}.

O cálculo do \textit{score} dos modelos foi realizado utilizando o coeficiente de determinação \textit{R\textsuperscript{2}}.
O coeficiente de determinação \textit{R\textsuperscript{2}} é a proporção de variância dos dados que pode ser explicada pelo modelo de regressão. Ela é utilizada para medir o sucesso da predição de uma variável com base em outras variáveis independentes e quanto mais próximo de 1 maior a acurácia do modelo \cite{NAGEKERKE1991}.

A função padrão de \textit{score} do \textit{Scikit-learn} retorna valores entre 0 e 1 para modelos onde a capacidade preditiva é melhor do que a média aritmética dos valores da amostra, ou seja, caso o modelo possua um \textit{score} inferior à 0 o cálculo simples da média dos valores da amostra seria mais preciso que o modelo treinado.

Os \textit{scores} obtidos para os modelos construídos são apresentados na tabela \ref{tab:score-1}, no qual o conjunto 1 representa o conjunto de dados que considera somente uma avaliação e o conjunto 2 leva em conta as duas avaliações.

\begin{table}[!htb]
    \centering
    \caption{Score dos modelos construídos considerando os conjuntos com uma avaliação e duas avaliações}
    \begin{tabular}{ccc}
        \textbf{} & \multicolumn{2}{c}{\textbf{Score}} \\ \hline
        \textbf{Algoritmo} & \textbf{Conjunto 1} & \textbf{Conjunto 2} \\
        Regressão Linear & 99,9\% & 99,7\% \\
        Gradient Boosting & 89,8\% & 97,4\% \\
        Random Forest & 89,4\% & 96,2\% \\ \hline
    \end{tabular}
    \label{tab:score-1}
\end{table}

Percebe-se que os modelos construídos com o algoritmo de regressão linear possuem acurácia muito elevada quando comparado aos outros algoritmos, principalmente para o conjunto 1. 
As tabelas \ref{tab:teste-1} e \ref{tab:teste-2} apresentam os resultados retornados para o conjunto de 20\% de testes para os modelos treinados. 

\begin{table}[!htb]
    \centering
    \caption{Predições realizadas sobre o subconjunto de testes para o conjunto de uma avaliação}
    \begin{tabular}{cccccccc}
        \hline
        \textbf{Algoritmo} & \textbf{1º} & \textbf{2º} & \textbf{3º} & \textbf{4º} & \textbf{5º} & \textbf{6º} & \textbf{7º} \\ \hline
        Gradient Boosting & 64,4 & 9,9 & 64,6 & 44,6 & 17,7 & 2,6 & 14,0 \\
        Random Forest & 65,5 & 10,3 & 67,4 & 42 & 17,6 & 6,6 & 14,6 \\
        Regressão Linear & 68,6 & 10,6 & 56,7 & 31,9 & 25,6 & -0,1 & 20,9 \\
        Valor real & 69,0 & 11,0 & 57,0 & 32 & 26,0 & 0,0 & 21,0 \\ \hline
    \end{tabular}
    \label{tab:teste-1}
\end{table}

\begin{table}[!htb]
    \centering
    \caption{Predições realizadas sobre o subconjunto de testes para o conjunto de duas avaliações}
    \begin{tabular}{ccccccccllllll}
        \hline
        \textbf{Algoritmo} & \textbf{1º} & \textbf{2º} & \textbf{3º} & \textbf{4º} & \textbf{5º} & \textbf{6º} & \textbf{7º} & \textbf{8º} & \textbf{9º} & \textbf{10º} & \textbf{11º} \\ \hline
        Gradient Boosting & 75,4 & 63,6 & 29,4 & 26,4 & 66,9 & 82,2 & 25,5 & 0,0 & 0,0 & 78,3 & 71,5 \\
        Random Forest & 81,3 & 60,5 & 27,1 & 25,7 & 63,4 & 76,2 & 26,2 & 0,6 & 0,1 & 78,8 & 71,4 \\
        Regressão Linear & 79,5 & 59,1 & 22,8 & 18,1 & 65,3 & 76,9 & 27,7 & -0,1 & 0,5 & 79,7 & 68,7 \\
        Valor real & 80,0 & 60,0 & 25,0 & 18,0 & 64,0 & 77,0 & 29,0 & 0,0 & 0,0 & 80,0 & 69,0 \\ \hline
    \end{tabular}
    \label{tab:teste-2}
\end{table}

Dentre os algoritmos, o \textit{Gradient Boosting} e \textit{Random Forest} são considerados mais adequados para um volume pequeno de dados de treinamento e mais robustos quanto a \textit{overfitting} quando comparados aos algoritmos de regressão linear.
Desta forma, a fim de verificar a acurácia dos modelos para diferentes situações foi-se executada a validação cruzada dos modelos, que divide o conjunto de dados em subconjuntos, treina e testa o modelo, retornando a acurácia média e desvio padrão dos diferentes subconjuntos.

A tabela \ref{tab:validacao-cruzada} apresenta a média das acurácias dos modelos e o desvio padrão entre as acurácias obtidas com o \textit{Scikit-learn}, considerando uma validação cruzada de três subconjuntos.

\begin{table}[!htb]
    \centering
    \caption{Scores médios e desvios padrões obtidos com a validação cruzada em três subconjuntos}
    \begin{tabular}{ccccc}
      \textbf{} & \multicolumn{2}{c}{\textbf{Conjunto 1}} &   \multicolumn{2}{c}{\textbf{Conjunto 2}} \\ \hline
        \textbf{Algoritmo} & \textbf{Score} & \textbf{D. Padrão} & \textbf{Score} & \textbf{D. Padrão} \\ \hline
        Gradient Boosting  & 95,8\% & $\pm 3,5$\% & 92,1\% & $\pm 0,9$\% \\
        Random Forest & 90,0\% & $\pm 5,6$\% & 88,6\% & $\pm 2,3$\% \\
        Regressão Linear & 85,8\% & $\pm 9,7$\% & 75,1\% & $\pm 0,7$\%  \\ \hline
    \end{tabular}
    \label{tab:validacao-cruzada}
\end{table}

Com a validação cruzada foi possível observar um melhor desempenho do algoritmo \textit{Gradient Boosting} para ambos os casos, retornando também um menor desvio padrão para o conjunto 1.
Dentre os três algoritmos, o \textit{Random Forest} obteve acurácias intermediárias e a regressão linear foi a que retornou o pior desempenho.

Devido ao algoritmo \textit{Gradient Boosting} apresentar os melhores resultados na maioria dos casos, este foi empregado como a técnica utilizada para retornar as predições em toda a plataforma desenvolvida neste trabalho.

\subsection{Importância}

As tabelas \ref{tab:importanciaGradient} e \ref{tab:importanciaRandom} mostram a importância das variáveis para os algoritmos \textit{Gradient Boosting} e \textit{Random Forest} para o conjunto de duas avaliações.
Nelas é possível observar que para o conjunto de dados utilizados a nota da segunda avaliação tem uma influência muito grande no resultado final das predições, possuindo importâncias de 79,9\% e 55,7\% para os dois modelos, respectivamente.

Os modelos divergem quanto a segunda variável com maior influência sobre a predição, mostrando que tanto a nota do primeiro trabalho quanto a nota da primeira avaliação podem ser importantes nesse contexto.

A frequência do aluno ficou em 4º lugar com 3,2\% na lista do primeiro algoritmo e em 3º lugar com 14,5\% para o segundo, apresentando certa diferença no nível de importância mas mostrando que em ambos os casos esta variável ainda se faz presente nas predições.

Em ambos os modelos as variáveis com menor importância foram as notas do segundo e terceiro trabalhos, com influências de 1,7\% ou menos nas estimativas.

Já a tabela \ref{tab:importanciaUmaProva} lista as variáveis e suas importâncias considerando alunos que só realizaram uma avaliação. Neste caso o nível de incerteza dependerá do padrão da disciplina. Caso o previsto seja a aplicação de três avaliações, o nível de incerteza se torna maior quanto à nota final dos alunos, pois há mais variáveis desconhecidas do que conhecidas para se realizar a predição, por exemplo, conhece-se a nota da primeira avaliação mas não da segunda e terceira avaliações.
Neste caso, pode-se verificar que a nota da primeira avaliação está diretamente relacionada a nota final prevista para o aluno, com 89,6\% e 92,7\% de importância, respectivamente, para as predições realizadas pelo \textit{Gradient Boosting} e pelo \textit{Random Forest}.

Ao comparar as tabelas \ref{tab:importanciaGradient} e \ref{tab:importanciaRandom} com a tabela \ref{tab:importanciaUmaProva} é possível observar que para o conjunto de dados que possui notas de duas avaliações a importância se diluiu mais entre as \textit{features} disponíveis, enquanto que para dados com somente uma avaliação a importância se concentrou na nota da avaliação realizada. 
Complementar a isto, os dados dos exercícios da plataforma URI tiveram uma participação muita baixa nas estimativas dos modelos, com cerca de 1,0\% de influência para ambos os algoritmos.

Em seguida tem-se a tabela \ref{tab:importanciaComDatas}, que mostra o resultado de um teste realizado considerando como \textit{features} cada um dos dias de aula.
Foi possível observar que individualmente as aulas não tiveram participação considerável na predição das notas finais dos alunos, com um máximo de 0,2\% por aula individual e total de 0,9\% ao somar as importâncias de todas as aulas do modelo construído com o \textit{Gradient Boosting}, que foi o modelo que retornou maior influência destas variáveis.

Por fim a tabela \ref{tab:coef-linear} mostra os coeficientes determinados para a equação linear gerada pelo algoritmo de regressão linear. 
E nesta é possível observar uma maior distribuição da influência entre as variáveis, apresentando, como os outros algoritmos, um coeficiente mais elevado para a a nota da segunda avaliação, seguida da nota da primeira avaliação, com 0,473 e 0,407, respectivamente, somando um coeficiente conjunto de 0,88 para as avaliações. 
Já para os trabalhos os coeficientes somados representam um valor de 0,116. 
O restante fica por conta da frequência e da porcentagem de exercícios do URI resolvidos. 
Este último obteve um coeficiente negativo, porém muito pequeno, o que se torna praticamente irrelevante para a estimativa da nota final.

\section{IMPLEMENTAÇÃO}

Os resultados obtidos acerca da plataforma desenvolvida foram mais subjetivos, mas é possível visualizar os potenciais que a ferramenta têm para o auxílio dos usuários na análise dos dados acadêmicos.

\subsection{Dashboard}

Com o dashboard implementado é possível extrair alguns \textit{insights} sobre os dados. 
O gráfico temporal apresentado na figura \ref{fig:dashboard-1} possibilita verificar a proporção de faltas da turma como um todo, e caso a linha de faltas da turma ultrapasse a linha que representa o limite permitido de faltas, o profissional pode tomar ações para compreender em detalhes quais os motivos que levam a esse acontecimento.
Já com os indicadores da parte superior da mesma página, pode-se ter uma visão geral das turmas cadastradas ou de uma turma específica selecionado-a na caixa de seleção flutuante presente à direita.

A partir do gráfico de dispersão do dashboard, figura \ref{fig:dashboard-2}, pode-se verificar visualmente o nível de correlação presente entre a frequência e as notas dos alunos, e caso a influência dos trabalhos na nota final seja explícita o profissional pode tomar ações ou apresentar este gráfico aos alunos, argumentando e influenciando-os a seguir o perfil que tende a receber melhores notas.

No mesmo gráfico pode-se verificar que há alunos com frequências próximas ou iguais a 100\% mas com notas abaixo da média para aprovação, desta forma é possível acessar estes alunos a fim de averiguar quais os motivos que os levam a não obterem desempenho tão bons na disciplina.
Com isso pode-se descobrir lacunas no método pedagógico utilizado pelo docente e oportunidades de utilização de recursos para a melhoria do aprendizado, como os alunos monitores com um maior foco nestes alunos.

Ainda no dashboard, o gráfico de barras permite uma análise rápida das notas do alunos nas disciplinas e a comparação entre elas, assim pode-se identificar a proporção de alunos com um bom, médio e baixo desempenho, apurando também as disciplinas com menor aproveitamento por conta dos alunos.

\subsection{Predição, Feedback e Mecanismos de Detecção}

A predição das notas dos alunos em conjunto com a detecção de situações incomuns complementam o sistema proposto e desenvolvido neste trabalho, que tem como objetivo fornecer ferramentas para que os profissionais possam analisar dados acadêmicos e tomar algumas ações simples sem a necessidade de muitos esforços manuais.

A página de detalhes do estudante proporciona ao docente uma visão da situação atual do aluno, permitindo nela comparar o aluno à média, tanto em faltas quanto em notas. 
Esta é uma forma de investigar um aluno a fim de identificar padrões no seu comportamento, como a diminuição das notas coincidentemente na época do período letivo que o aluno mais obteve faltas.

Ainda na mesma página o profissional poderá submeter uma notificação ao aluno, solicitando um \textit{feedback}. 
Este recurso foi idealizado para o uso em situações que demandam uma advertência, mas ele não é restrito a isso. 
Caso o docente observe alguma variação no perfil dos alunos, este pode pedir um \textit{feedback} também em uma situação ainda considerada adequada para o aluno.
Um exemplo de padrão que o professor pode identificar é a ausência de determinados alunos no mesmo dia da semana. 
Deduz-se então que algum evento em especial pode estar ocorrendo nestes dias e o docente pode enviar um e-mail aos alunos a fim de comprovar seu \textit{insight}.

Um dos objetivos almejados com a possibilidade de os alunos enviarem \textit{feedbacks} é verificar se a comunicação com os alunos pode ser uma forma de conscientização e consequentemente melhores resultados no desempenho dos alunos.

Observou-se com a base de dados utilizada neste trabalho que existem diferentes perfis de estudantes e este sistema é destinado a atingir a parcela de alunos que possui dedicação aos estudos mas por algum motivo sob o domínio do professor não consegue obter um bom desempenho e devido a isto possui perfil tendencioso a desistência ou reprovação nas disciplinas.

Como descrito anteriormente, o algoritmo selecionado para a predição das notas dos alunos foi o \textit{Gradient Boosting}.
Com ela utilizando uma base histórica tornou-se possível estimar o desempenho dos alunos, a fim de tomar ações e inibir a retenção acadêmica antes do seu acontecimento.

Em contrapartida observou-se que o algoritmo de regressão linear determinou o coeficientes da sua equação linear com valores muito semelhantes aos pesos para avaliações e trabalhos definidos no carregamento dos dados na ferramenta.
Os pesos utilizados no momento do carregamento foram 0,9 para as avaliações e 0,1 para os trabalhos e os coeficientes retornados pelo algoritmo para as avaliações e para os trabalhos foram 0,880 e 0,116, respectivamente, o que comprovou uma boa capacidade de modelar o método de avaliação por conta do algoritmo.
Para dados que não seguiam um comportamento linear o \textit{Gradient Boosting} se mostrou melhor.

O sistema construído segmenta os alunos em três perfis, o primeiro com mais chances de aprovação na disciplina, o segundo com estimativas medianas e o terceiro com maiores riscos de reprovação. 
Com isso o profissional pode selecionar um perfil específico e atuar sobre os alunos contidos nele com o intuito de melhorar suas situações e colaborar com a transição destes para perfis mais positivos.