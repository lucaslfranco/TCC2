% RESULTADOS-------------------------------------------------------------------

\chapter{CRONOGRAMA}

O cronograma foi construído com base no método CRISP-DM, apresentado na seção \ref{ssec:crisp}, que será utilizado para definir a sequência de tarefas que serão realizadas durante o período de trabalho. 

Devido a natureza iterativa do modelo CRISP-DM, foram acrescentados períodos de trabalho extra para as atividades que requerem revisão e ajustes durante o desenvolvimento do projeto.
Como exemplo de tarefas tem-se a \textit{definição de objetivos de negócio}, que podem sofrer alterações conforme o resultado obtido na tarefa de \textit{análise dos dados}. 
De forma semelhante ocorre a \textit{seleção das técnicas de mineração de dados}, que dependerão da assertividade apresentada pelas técnicas com o conjunto de dados utilizado no trabalho.

O cronograma é apresentado à seguir, na \autoref{tab:cronograma}.

\begin{landscape}
\begin{table}[!htb]
    \centering
    \caption[Cronograma]{Cronograma}
    \tiny
    \label{tab:cronograma}
    \begin{tabular}{*{40}{c}}
        \topline
          \\
              \multicolumn{2}{l}{}
            & \multicolumn{4}{c|}{Novembro}
            & \multicolumn{4}{c|}{Dezembro} 
            & \multicolumn{4}{c|}{Fevereiro} 
            & \multicolumn{4}{c|}{Março} 
            & \multicolumn{4}{c|}{Abril} 
            & \multicolumn{4}{c|}{Maio} \\
            
        \hline
          \multicolumn{2}{l}{}
            & 1 & 2 & 3 & 4 & 
            1 & 2 & 3 & 4 &
            1 & 2 & 3 & 4 &
            1 & 2 & 3 & 4 &
            1 & 2 & 3 & 4 &
            1 & 2 & 3 & 4 &\\
            
        \midrule
            \multirow{2}{*}{\begin{tabular}[c]{@{}l@{}} Levantamento \\ bibliográfico\end{tabular}} 
            & Estudo de Técnicas de MD      & X & X & & & \\\cmidrule(r){2-26}
            & Estudo de Ferramentas         & & X & X & & \\\cmidrule(l){1-26}
            
            \multirow{2}{*}{\begin{tabular}[c]{@{}l@{}} Entendimento \\ do negócio \end{tabular}}
            & Definir objetivos de negócio  & & & X & X & & & & & & X &\\\cmidrule(r){2-26}
            & Definir objetivos de MD       & & & & X & X & & & & & & X &   \\\cmidrule(l){1-26}
         
            \multirow{3}{*}{\begin{tabular}[c]{@{}l@{}} Entendimento \\ dos dados \end{tabular}} 
            & Scripts Coleta de dados  & & & & X & X & \\\cmidrule(r){2-26}
            & Coleta inicial dos dados      & & & & & X & X & &   \\\cmidrule(r){2-26}
            & Análise dos dados             & & & & & & X & X & & & & & & & X & X & \\\cmidrule(l){1-26}
            
            \multirow{3}{*}{\begin{tabular}[c]{@{}l@{}} Preparação \\ dos dados \end{tabular}} 
            & Scripts Prep. dos dados           & & & & & & X & X & X & \\\cmidrule(r){2-26}
            & Seleção dos dados                 & & & & & & & X & & & & & & & & & X & \\\cmidrule(r){2-26}
            & Limpeza e construção              & & & & & & & & X & X & & & & & & & & X &\\\cmidrule(r){2-26}
            & Integração e Formatação           & & & & & & & & & X & X & & & & & & & & X &\\\cmidrule(l){1-26}
            
            \multirow{4}{*}{\begin{tabular}[c]{@{}l@{}} Modelagem \end{tabular}}
            & Seleção de técnicas de MD     & & & & & & & & & X & & & & & & & & X & \\\cmidrule(r){2-26}
            & Construção dos modelos        & & & & & & & & & & X & X & X & & & & & & & X & X & \\\cmidrule(r){2-26}
            & Testes empíricos              & & & & & & & & & & & & X & X & X & & & & & & X & X &\\\cmidrule(l){1-26}

            \multirow{3}{*}{\begin{tabular}[c]{@{}l@{}} Validação \\ dos modelos \end{tabular}}
            & Avaliação dos objetivos       & & & & & & & & & & & & & & & X & X & & & & & X & \\\cmidrule(r){2-26}
            & Avaliação da acurácia         & & & & & & & & & & & & & & & X & X & & & & & X & \\\cmidrule(r){2-26}
            & Revisão dos modelos           & & & & & & & & & & & & & & & & & X \\\cmidrule(l){1-26}
            
            \multirow{2}{*}{\begin{tabular}[c]{@{}l@{}} Implantação \end{tabular}}
            & Sistema web                   & & & & & & & & & & & & & & & X & X & X & X & X & \\\cmidrule(r){2-26}
            & Apresentação dos dados           & & & & & & & & & & & & & & & & X & X & X & X & X & \\\cmidrule(r){2-26}
            & Mecanismos automatizados      & & & & & & & & & & & & & & & & & & & X & X & X & X \\\cmidrule(l){1-26}
         
            \multirow{1}{*}{\begin{tabular}[c]{@{}l@{}} Conclusão \end{tabular}}
            &               & & & & & & & & & & & & & & & & & & & & & & & X & \\
    
        \bottomrule
    \end{tabular}
\end{table}
\end{landscape}

