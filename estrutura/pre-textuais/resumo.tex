% RESUMO--------------------------------------------------------------------------------

\begin{resumo}[RESUMO]
\begin{SingleSpacing}

% Não altere esta seção do texto--------------------------------------------------------
\imprimirautorcitacao. \imprimirtitulo. \imprimirdata. \pageref {LastPage} f. \imprimirprojeto\ – \imprimirprograma, \imprimirinstituicao. \imprimirlocal, \imprimirdata.\\
%---------------------------------------------------------------------------------------

A retenção no ensino superior é um fenômeno complexo que causa grandes impactos na sociedade, devido a todos os fatores que estão intrinsecamente relacionados com sua ocorrência, como o financeiro, educacional e social.
Este trabalho tem a retenção como objeto de estudo e apresenta a construção de um sistema para análise de dados acadêmicos utilizando técnicas de mineração de dados.
Em conjunto são implementados mecanismos automáticos para comunicação com alunos e artifícios para que docentes possam extrair \textit{insights} e tomar ações a fim de contribuir para a redução da retenção acadêmica.
\\

\textbf{Palavras-chave}:
Aprendizado de máquina. Ensino superior. Reconhecimento de padrões. Retenção acadêmica. Sistema acadêmico.

\end{SingleSpacing}
\end{resumo}

% OBSERVAÇÕES---------------------------------------------------------------------------
% Altere o texto inserindo o Resumo do seu trabalho.
% Escolha de 3 a 5 palavras ou termos que descrevam bem o seu trabalho 
