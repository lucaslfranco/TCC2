\chapter{INTRODUÇÃO}
\label{chap:introducao}

A evasão e a retenção acadêmica no ensino superior são fenômenos que possuem várias vertentes e trazem muitas discussões. Isto porque envolve fatores que não estão somente relacionados ao ensino e a educação, mas à outros tópicos importantes para a sociedade, como o viés econômico. Segundo \citeonline{SilvaFilho2007}, fundadores do Instituto Lobo, que tem como objetivo contribuir com soluções para os problemas brasileiros relacionados às áreas de educação, ciência e tecnologia, o desperdício de recursos devido a evasão tanto em Instituições de Ensino Superior (IES) públicas quanto privadas é de mais de dez bilhões de reais anualmente, e mesmo que seja um tema discutido, não recebe a devida importância e ações necessárias.

\citeonline{SilvaFilho2007} apresentam também, baseado em estudos nacionais e internacionais feitos pelo Instituto Lobo, uma lista de ações consideradas bem-sucedidas para o combate e redução da evasão nas IES brasileiras. A seguir são apresentadas algumas destas ações.

\begin{enumerate}
    \item Análise estatística do fenômeno da evasão;
    \item Criação de programas de aconselhamento e orientação de alunos;
    \item Envolvimento da comunidade acadêmica no combate à evasão estimulando a visão centrada no aluno;
    \item Criação de condições que satisfaçam os objetivos responsáveis por atrair os alunos às IES e apresentar casos positivos de satisfação;
    \item Modernização da forma como são ministrados os cursos:
    \item Introdução de atividades relacionadas à profissões desde o início do curso;
    \item Estímulo para o desenvolvimento de atitudes inovadoras e empreendedoras pelos estudantes.
\end{enumerate} 

Além das ações consideradas convencionais apresentadas, existem também ações tecnológicas, que podem ser aplicadas a fim de torná-las mais eficientes, como por exemplo a implementação de Sistemas Inteligentes. 
Desenvolvidos respeitando as especificidades de cada instituição, estes sistemas podem auxiliar no acompanhamento do desempenho de alunos, professores e cursos, possibilitando a identificação de situações "críticas", como a de alunos que apresentam grande possibilidade de reprovação ou evasão, dificuldades pedagógicas por conta de professores, disciplinas com altos índices de reprovação e os possíveis motivos para estas ocorrências.

Das ações apresentadas, a número 1 é a que mais se enquadra ao escopo deste trabalho, que visou, a partir da análise de dados acadêmicos, compreender a retenção no ensino superior, descrever estratégias que possibilitem predizer a situação de alunos e dar suporte para que outros grupos (pedagogos, assistentes estudantis e professores) usufruam e tomem ações a partir delas. 

\section{PROBLEMA}
\label{sec:problema}

Evidencia-se em números a problemática que se inserem as IES brasileiras quando trata-se da \textbf{evasão} e da \textbf{retenção}. Temas delicados devido a natureza subjetiva dos motivos que levam ao acontecimento, os índices de desistência e reprovação no Ensino Superior cresceu consideravelmente nos últimos anos, apresentando um aumento de mais de 37\% entre os anos de 2010 e 2014, segundo o Censo da Educação Superior 2015, divulgado pelo \citeonline{INEP2016}. 

Segundo \citeonline{Santos2018}, o número de vagas ociosas nas instituições federais brasileiras é de mais de 25\% do total de vagas ofertadas, devido à evasão e ao não preenchimento destas em processos seletivos. Nas instituições estaduais estes números aproximam-se de 17\%.

\citeonline{SilvaFilho2007} argumentam em seu trabalho que essa ociosidade de vagas na educação superior gera, além da perda da qualificação profissional dos estudantes, impactos econômicos negativos para a nação como um todo, uma vez que, os recursos públicos que foram aplicados para a criação destas vagas não surtiram um retorno correspondente.

E, como destacado por \citeonline{Marques2015}, o contexto de dados acadêmicos pode envolver um considerável volume de dados, uma vez que este pode ser escalável e evoluir para diferentes instituições de ensino.
Além do volume, tem-se a substancial variedade dos dados, já que cada docente possui certa autonomia para estruturar as disciplinas vinculadas da forma que lhe convém, considerando ainda diferentes instituições de ensino, com diferentes padrões de avaliação.

\section{JUSTIFICATIVA}
\label{sec:justificativa}
Devido a atual situação das IES em relação as taxas de retenção, se faz necessário a criação de ferramentas que tornem possível agir sobre a retenção em cursos de graduação, reduzindo estes números. Além de traçar perfis que permitam compreender o comportamento dos alunos e predizer possíveis situações a partir da análise de dados acadêmicos prévios. 

\section{OBJETIVOS}
\label{sec:objetivos}

O objetivo geral deste trabalho é a análise de dados acadêmicos e o desenvolvimento de uma plataforma utilizando técnicas de mineração de dados, a fim de permitir uma análise mais profunda dos dados acadêmicos por conta dos usuários, permitindo assim extrair \textit{insights} e construir mecanismos de apoio à redução da retenção no ensino superior.

\subsection{Objetivos Específicos}

\begin{itemize}
    \item Construção de gráficos para fácil apresentação e análise de dados;
    \item Implementação de algoritmos de aprendizado de máquina e reconhecimento de padrões que atuem a fim de identificar e prever a retenção em cursos de graduação; 
    \item Desenvolvimento de um sistema web para que docentes tenham acesso aos dados acadêmicos, gráficos e mecanismos;    
    \item Desenvolvimento de mecanismos automatizados que possibilitem a comunicação com alunos em determinadas situações, como a situação de risco de reprovação por faltas em determinada disciplina.
\end{itemize}

\section{ORGANIZAÇÃO DO TRABALHO}
\label{sec:organizacaoTrabalho}

Este trabalho é dividido em cinco capítulos. 
O capítulo dois contém a fundamentação teórica, apresentando a contextualização de temas pertinentes ao trabalho e a revisão da literatura realizada. 
O capítulo três descreve a proposta de trabalho, destacando as tecnologias e ferramentas utilizadas, as etapas executadas com o processo metodológico utilizado e o detalhamento da implementação efetuada apresentando cada uma das partes do sistema desenvolvido.
O capítulo seguinte apresenta a análise dos resultados, levando em conta os modelos de predição e o sistema implementado. 
E o último capítulo contém as considerações finais apresentadas para este trabalho.